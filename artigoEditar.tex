%--------------------------------------------------------------------------------------------------
% OBSERVACAO:
% 
% -> Arquivos que você pode editar:
%    - artigo.tex
%    - artigo_bibliografia.bib
%
% -> Arquivo .TeX codificado em UTF8                                                             
% -> Bibliografia em arquivo .bib (arquivo_bibliografia.bib)                                      
% -> Arquivo de imagens em .jpg, .eps ou .pdf
% -> Para compilar o TeX, execute 'compila_TEX.bat' (terminal do windows)
% 
% versão 1.1 - 19/05/2016
% versão 1.0 - 18/08/2015
%--------------------------------------------------------------------------------------------------
\documentclass{classe_cn}                 % Modelo <nao edite o arquivo classe_cn.cls>
\usepackage[brazil]{babel}                % Acentos
\usepackage[utf8]{inputenc}               % Codificação UTF8 (atenção aqui!)
\usepackage{graphicx}                     % Figura
\usepackage{amssymb}                      % Simbolos matematicos
\usepackage{color}                        % Cores
\usepackage{amsfonts}                     % Fontes
\usepackage{amsmath}                      % Fontes
\usepackage[fixlanguage]{babelbib}        % Acentos
\usepackage[normalem]{ulem}               % OK
\usepackage[retainorgcmds]{IEEEtrantools} % Formulas padrão IEEE
\usepackage{omlmathbf}                    % Simbolos Matematicos
\usepackage{epstopdf}                     % Figuras .eps
\usepackage{setspace}                     % Espaçamento flexível
\usepackage{cmap}                         % Mapear caracteres especiais no PDF
\usepackage{textcomp}                     % Funções e outros símbolos matemáticos
\usepackage{verbatim}                     % Pacotes verbatim
\usepackage{wrapfig}
\usepackage{picins}
\startlocaldefs
\endlocaldefs

%--------------------------------------------------------------------------------------------------
% Inicio do Documento
%--------------------------------------------------------------------------------------------------
\begin{document}
\begin{frontmatter}        % Não alterar
\begin{fmbox}              % Não alterar
\dochead{Gerência da informação} % Não alterar

%--------------------------------------------------------------------------------------------------
% Titulo do seu Trabalho
%   - pequeno bug (nao funciona cedilha)
%   - editar manualmente o cedilha na classe_cn.cls, linha 1015.
%--------------------------------------------------------------------------------------------------
\title{Internet das coisas}

%------------------------------------------------
% Informações sobre o autor #1
% - Alice Fernandes Silva
%------------------------------------------------
\author[
  addressref = {aff1},                 % Identifica o autor #1
  email      = {alice.silva@ccc.ufcg.edu.br} % email para contato
]
{
  \inits{AFS}      % Letras iniciais do autor #1
  \fnm{Alice F.}  % Nome do autor #1 (first and middle name)
  \snm{Silva}   % Ultimo nome do autor #1 (last name)
}
%------------------------------------------------
% Informações sobre o autor #2
% - Ellen
%------------------------------------------------
\author[
  addressref = {aff1},                      % Identifica o autor
  email      = {ellen@ccc.ufcg.edu.br} % email para contato
]
{
  \inits{E}       % Letras iniciais do autor #2
  \fnm{Ellen}  % Nome do autor #2 (first and middle name)
  \snm{Desconhecido ainda}    % Ultimo nome do autor #2 (last name)
}
%------------------------------------------------
% Informações sobre o autor #3
% - Ivyna Rayany Santino Alves
%------------------------------------------------
\author[
  addressref = {aff1},                       % Identifica o autor
  email      = {ivyna.alves@ccc.ufcg.edu.br} % email para contato
]
{
  \inits{IRSA}      % Letras iniciais do autor #3
  \fnm{Ivyna Rayany Santino} % Nome do autor #3 (first and middle name)
  \snm{Alves}    % Ultimo nome do autor #3 (last name)
}
%------------------------------------------------
% Informações sobre o autor #4
% - Kaio Kassiano Moura Oliveira 
%------------------------------------------------
\author[
  addressref = {aff1},                 % Identifica o autor
  email      = {kaio.kassiano.oliveira@ccc.ufcg.edu.br} % email para contato
]
{
  \inits{KKMO}     % Letras iniciais do autor #4
  \fnm{Kaio Kassiano Moura} % Nome do autor #4 (first and middle name)
  \snm{Oliveira}     % Ultimo nome do autor #4 (last name)
}

%------------------------------------------------
% Endereço dos autores
%------------------------------------------------
\address[id=aff1]{
  \orgname{Universidade Federal de Campina Grande,
           Centro de Engenharia Elétrica e Informática,
           Unidade Acadêmica de Sistemas e Computação},
  \street{Rua Aprígio Veloso, 882, Bairro Universitário},
  \postcode{58429-140},
  \city{Campina Grande},
  \cny{Brasil.}
}

\end{fmbox}

%--------------------------------------------------------------------------------------------------
% Resumo do Trabalho
%--------------------------------------------------------------------------------------------------
\begin{abstractbox}
	
\begin{abstract} 
Escrever no máximo $150$ palavras no resumo do trabalho. Exemplo: The objective of this work is to determine if people are interacting in TV video by detecting whether they are looking at each other or not.We determine both the temporal period of the interaction and also spatially localize the relevant people. We make the following four contributions: (\textit{i}) head detection with implicit coarse pose information (front, profile, back); (\textit{ii}) continuous head pose estimation in unconstrained scenarios (TV video) using Gaussian process regression; (\textit{iii}) propose and evaluate several methods for assessing whether and when pairs of people are looking at each other in a video shot; and (\textit{iv}) introduce new ground truth annotation for this task, extending the TV human interactions dataset. The performance of the methods is evaluated on this dataset, which consists of $300$ video clips extracted from TV shows. Despite the variety and difficulty of this video material, our best method obtains an average precision of $87.6\%$ in a fully automatic manner.
\end{abstract}

%--------------------------------------------------------------------------------------------------
% Palavras-chaves: Entre 3 e 6 palavras chaves
%--------------------------------------------------------------------------------------------------
\begin{keyword}
  \kwd{Escreva}
  \kwd{algumas}
  \kwd{palavras-chaves}
  \kwd{aqui!}
\end{keyword}

\end{abstractbox} % Não alterar
\end{frontmatter} % Não alterar

%--------------------------------------------------------------------------------------------------
% Escreva o seu artigo!
%--------------------------------------------------------------------------------------------------

%------------------------------------------------
% Seção 1
%------------------------------------------------
\section{Introdução}

Escreva introdução e motivação do seu trabalho. Tente convencer o leitor da importância da sua pesquisa. Exemplo: If you read any book on film editing or listen to a director's commentary on a DVD, then what emerges again and again is the importance of eyelines. Standard cinematography practice is to first establish which characters are looking at each other using a medium or wide shot, and then edit subsequent close-up shots so that the eyelines match the point of view of the characters. This is the basis of the well known $180^{o}$ rule in editing.

The objective of this paper is to determine whether eyelines match between characters within a shot—and hence understand which of the characters are interacting~\cite{Pressman:2007}. The importance of the eyeline is illustrated by the three examples of Figure~\ref{tag_figura_01} - one giving rise to arguably the most famous quote from Casablanca, and another being the essence of the humour at that point in an episode of Fawlty Towers. Our target application is this type of edited TV video and films. It is very challenging material as there is a wide range of human actors, camera viewpoints and ever present background clutter. The thirty Brodatz textures u asl sklsksj slk slk dsed are shown in Figure~\ref{tag_figura_01}.

begin
\begin{figure}[h!]
  \begin{center}
    \includegraphics[width=1.0 \textwidth]{figura01.jpg}
    \caption{Exemplo de Figura 01.} 
    \label{tag_figura_01}
  \end{center}
\end{figure}

Gray level co-occurrence matrix (GLCM)~\cite{Ferris:2003} describes the relative frequencies with which two pixels separated by a distance $d$ under a specified angle occur on the image. Then, the GLCM matrices are pre-processed in order to obtain input data for the clustering or classification modules. In the Clustering module the SOM neural network organises and extracts prototypes from the processed matrices, which ends the learning stage. The classification module receives a pre-processed query image and compares it with the prototypes (representations of clusters) obtained in the clustering module. The final result is a list of images belonging to a few number of clusters considered to be the nearest to the user's query. Figure~\ref{tag_figura_02} shows these building blocks.

\begin{figure}[h!]
  \begin{center}
    \includegraphics[width=1.0 \textwidth]{figura02.jpg}
    \caption{Exemplo de Figura 02.} 
    \label{tag_figura_02}
  \end{center}
\end{figure}

%------------------------------------------------
% Seção 2
%------------------------------------------------
\section{Motivação}

If we assume that sensitive cells follow a deterministic decay $Z_0(t) = xe^{\lambda_0 t}$ and approximate their extinction time as $T_x \approx \frac{1}{\lambda_0} \log x$, then we can heuristically estimate the expected value as:

\begin{eqnarray}
\label{eqexpmuts}
  E [Z_1(vT_x)] &=& \frac{\mu}{r}\log x \int_0^{1} x^{1-u} du \\
  E [Z_1(vT_x)] &=& \frac{\mu}{r}x^{1-{\lambda_1}/{\lambda_0}v}\log  \\
  1 &=& 10
\end{eqnarray}

\begin{equation}
  E [Z_1(vT_x)] = \frac{\mu}{r}\log x \int_0^{1} x^{1-u} du \\
  E [Z_1(vT_x)] = \frac{\mu}{r}x^{1-{\lambda_1}/{\lambda_0}v}\log 
\end{equation}

Thus we observe that this expected value is finite for all $v>0$ (also see \cite{Rosenfeld:1970}).

%------------------------------------------------
% Sub-seção
%------------------------------------------------
\subsection{Definição} %Definição de IoT
É um conceito criado para a revolução tecnológica dos itens interconectados, 
com o objetivo de facilitar a realização de tarefas cotidianas.\\

“Mas o que eu quis dizer à época, e ainda considero isso válido, se baseia 
na ideia de que estamos presenciando o momento em que duas redes distintas – 
a rede de comunicações humana (exemplificada na internet) e o mundo real das coisas – 
precisam se encontrar. Um ponto de encontro onde não mais apenas "usaremos um 
computador", mas onde o "computador se use" independentemente, de modo a tornar a
vida mais eficiente. Os objetos – as "coisas" – estarão conectados entre si e em rede, 
de modo inteligente, e passarão a "sentir" o mundo ao redor e a interagir.”\\  

\textit{Kevin Ashton, pesquisador britânico do Massachusetts Institute of Technology(MIT), 
é considerado o primeiro especialista a usar o termo "Internet das Coisas" (IoT, na sigla em inglês), em 1999.}\\

Dessa forma, a IoT carrega a dinâmica e a interatividade às pessoas a partir 
da conexão e mobilidade dos dispositivos. 

\subsection{Surgimento do termo} %Surgimento de IoT
O termo “Internet das Coisas” é uma adaptação em português para a expressão em 
inglês “Internet of Things” (IoT, sigla em inglês), utilizada pela primeira 
vez pelo pesquisador britânico Kevin Ashton do Massachusetts Institute of 
Technology(MIT) no ano de 1999. Para Kevin, a revolução tecnológica iria 
proporcionar aos humanos uma maior independência dos aparelhos ao realizar 
tarefas sem a interferência direta humana.\\

“Numa apresentação para executivos da Procter & Gamble em 1999, quando eu 
falava da ideia de se etiquetar eletronicamente os produtos da empresa, 
para facilitar a logística da cadeia de produção, através de identificadores 
de radiofrequência (RFID, em inglês), na época um assunto novíssimo e "quente". 
A expressão "Internet das Coisas" pode nem ser tão brilhante, mas deu um bom 
título à apresentação, e logo se popularizou. Na verdade, a combinação de 
palavras foi como o resultado de um insight importante, de algo que ainda 
é mal compreendido.”\\

\textit{Kevin Ashton, pesquisador britânico do Massachusetts Institute of Technology(MIT).}\\

Um exemplo prático foi na preparação de atletas brasileiros na modalidade de canoagem, 
em parceria com a empresa multinacional GE desenvolveram um projeto de monitoramento 
dos movimentos dos esportistas, com objetivo de obter melhorias das performances nos 
treinos e, consequentemente, alcançar bons resultados nas Olimpíadas de 2016 na 
cidade do Rio de Janeiro. Este processo ocorreu através da captura dos dados de 
cada atleta no treinamento e enviados para a nuvem para técnicos analisarem os 
movimentos e terem o conhecimento da deficiência de cada atleta e trabalhar no 
aperfeiçoamento do rendimento de cada um.


\subsection{Como funciona}
Imagine uma situação, você entra cansado do trabalho, sua intenção 
é dormir,certo? Digamos que sua rotina geralmente é tomar um banho, 
jantar e depois dormir. Com a IoT, podemos espalhar sensores pela 
casa e com isso detectar seus movimentos do cotidiano e guardá-los 
em um banco de dados, dessa forma, os dispositivos que processam 
essas informações saberão sua rotina e passarão realizar tarefas 
e tomar decisões em situações do dia a dia.\\

Portanto, para a situação citada acima, os dispositivos conectados 
saberá a temperatura que você gosta de tomar banho. Após o banho, 
saberá que poderá jantar e sua geladeira ou armário informe se 
possui itens necessários para o preparo da comida. Após isso, 
saberá que talvez irá para o quarto e dormir, dessa forma irá 
deixar o quarto na temperatura ideal, a televisão ligará no canal 
mais assistido ou do programa que mais gosta até que durma e seu 
abajur e televisão desligue.\\

Mas como isso é possível? A conectividade é a palavra chave para o 
funcionamento dos aparelhos com o princípio de Internet das Coisas, 
logo torna possível a conexão inteligente entre os dispositivos e 
serem capazes de tomar decisões para a melhoria e conforto de 
atividades realizadas no cotidiano, seja ela como enviar mensagens 
quando não podemos atender, seja avisar que sua dispensa está ficando vazia.


%------------------------------------------------
% Exemplo de Tabela
%------------------------------------------------
\section{Exemplo de Tabela}

Table~\ref{tag_tabela_01} shows the average $ \alpha $ and the standard deviation for the CCR \cite{Rosenfeld:1970} obtained by the \textit{GLCM+SOM} method. We can conclude that for the Brodatz dataset~\cite{Domingues:2010} the processing tool based on mean vectors is the best option~\cite{Rosenfeld:1970, Diday:1989}. Considering this result~\cite{Visible:2013}, the mean vector approach is adopted as processing tool of the \textit{GLCM+SOM} method for the next experiments \cite{Fulano:2009}.

% Use a ferramenta para criar tabelas: http://www.tablesgenerator.com/
\begin{table}[h!]
\label{tag_tabela_01}
\caption{Sample table title. This is where the description of the table should go.}
  \begin{tabular}{cccc}
  \hline
       & B1   & B2   & B3   \\ \hline
   A1  & 0.1  & 0.2  & 0.3  \\
   A2  & ...  & ..   & .    \\
   A3  & ..   & .    & .    \\ \hline
  \end{tabular}
\end{table}

%--------------------------------------------------------------------------------------------------
%--------------------------------------------------------------------------------------------------
% Define o arquivo BIB (bibliografia)
%--------------------------------------------------------------------------------------------------
%--------------------------------------------------------------------------------------------------
\bibliographystyle{bmc-mathphys}   % NAO EDITAR!
\bibliography{artigo_bibliografia} % NAO EDITAR! - Bibliography file (usually '*.bib' )

\vspace{1.0cm}
\parpic{\includegraphics[width=1.5in,clip,keepaspectratio]{tesla.jpg}}
\noindent {\bf Fulano de Tal} was born in India. She received the B.S. 
degree in computer science from Kurukshetra University, Kurukshetra, 
India and the M.Phil. and Ph.D. degrees from the University of Exeter, 
Exeter, UK in 1999, 2001 and 2004, respectively. Her Ph.D. was in the 
area of machine learning for image analysis in aviation security. Her 
main research interests include image processing, natural scene analysis,
video analysis, and neural networks. She has published more than 30 papers
in the area of machine learning for image analysis in peer reviewed 
journals and conferences. Currently she is a Senior Research Fellow at
Loughborough University leading the project on imaging for road transport
applications.

\parpic{\includegraphics[width=1.5in,clip,keepaspectratio]{tesla.jpg}}
\noindent {\bf Fulano de Tal} was born in India. She received the B.S. 
degree in computer science from Kurukshetra University, Kurukshetra, 
India and the M.Phil. and Ph.D. degrees from the University of Exeter, 
Exeter, UK in 1999, 2001 and 2004, respectively. Her Ph.D. was in the 
area of machine learning for image analysis in aviation security. Her 
main research interests include image processing, natural scene analysis,
video analysis, and neural networks. She has published more than 30 papers
in the area of machine learning for image analysis in peer reviewed 
journals and conferences. Currently she is a Senior Research Fellow at
Loughborough University leading the project on imaging for road transport
applications.   


%\end{tabular}
%\end{table}

%--------------------------------------------------------------------------------------------------
% FIM DO ARTIGO
%--------------------------------------------------------------------------------------------------
\end{document}
