%--------------------------------------------------------------------------------------------------
% OBSERVACAO:
% 
% -> Arquivos que você pode editar:
%    - artigo.tex
%    - artigo_bibliografia.bib
%
% -> Arquivo .TeX codificado em UTF8                                                             
% -> Bibliografia em arquivo .bib (arquivo_bibliografia.bib)                                      
% -> Arquivo de imagens em .jpg, .eps ou .pdf
% -> Para compilar o TeX, execute 'compila_TEX.bat' (terminal do windows)
% 
% versão 1.1 - 19/05/2016
% versão 1.0 - 18/08/2015
%--------------------------------------------------------------------------------------------------
\documentclass[11pt]{classe_cn}                 % Modelo <nao edite o arquivo classe_cn.cls>
\makeatletter
\setattribute{leftarea}{width}{123\p@}
\makeatother
\usepackage[brazil]{babel}                % Acentos
\usepackage[utf8]{inputenc}               % Codificação UTF8 (atenção aqui!)
\usepackage{graphicx}                     % Figura
\usepackage{amssymb}                      % Simbolos matematicos
\usepackage{color}                        % Cores
\usepackage{amsfonts}                     % Fontes
\usepackage{amsmath}                      % Fontes
\usepackage[fixlanguage]{babelbib}        % Acentos
\usepackage[normalem]{ulem}               % OK
\usepackage[retainorgcmds]{IEEEtrantools} % Formulas padrão IEEE
\usepackage{omlmathbf}                    % Simbolos Matematicos
\usepackage{epstopdf}                     % Figuras .eps
\usepackage{setspace}                     % Espaçamento flexível
\usepackage{cmap}                         % Mapear caracteres especiais no PDF
\usepackage{textcomp}                     % Funções e outros símbolos matemáticos
\usepackage{verbatim}                     % Pacotes verbatim
\usepackage{wrapfig}
\usepackage{picins}
\startlocaldefs
\endlocaldefs

%--------------------------------------------------------------------------------------------------
% Inicio do Documento
%--------------------------------------------------------------------------------------------------
\begin{document}
\begin{frontmatter}        % Não alterar
\begin{fmbox}              % Não alterar
\dochead{Gerência da informação} % Não alterar

%--------------------------------------------------------------------------------------------------
% Titulo do seu Trabalho
%   - pequeno bug (nao funciona cedilha)
%   - editar manualmente o cedilha na classe_cn.cls, linha 1015.
%--------------------------------------------------------------------------------------------------
\title{Internet das coisas}

%------------------------------------------------
% Informações sobre o autor #1
% - Alice Fernandes Silva
%------------------------------------------------
\author[
  addressref = {aff1},                 % Identifica o autor #1
  email      = {alice.silva@ccc.ufcg.edu.br} % email para contato
]
{
  \inits{AFS}      % Letras iniciais do autor #1
  \fnm{Alice F.}  % Nome do autor #1 (first and middle name)
  \snm{Silva}   % Ultimo nome do autor #1 (last name)
}
%------------------------------------------------
% Informações sobre o autor #2
% - Ellen
%------------------------------------------------
\author[
  addressref = {aff1},                      % Identifica o autor
  email      = {ellen.oliveira@ccc.ufcg.edu.br} % email para contato
]
{
  \inits{EO}       % Letras iniciais do autor #2
  \fnm{Ellen}  % Nome do autor #2 (first and middle name)
  \snm{Oliveira}    % Ultimo nome do autor #2 (last name)
}
%------------------------------------------------
% Informações sobre o autor #3
% - Ivyna Rayany Santino Alves
%------------------------------------------------
\author[
  addressref = {aff1},                       % Identifica o autor
  email      = {ivyna.alves@ccc.ufcg.edu.br} % email para contato
]
{
  \inits{IRSA}      % Letras iniciais do autor #3
  \fnm{Ivyna Rayany Santino} % Nome do autor #3 (first and middle name)
  \snm{Alves}    % Ultimo nome do autor #3 (last name)
}
%------------------------------------------------
% Informações sobre o autor #4
% - Kaio Kassiano Moura Oliveira 
%------------------------------------------------
\author[
  addressref = {aff1},                 % Identifica o autor
  email      = {kaio.kassiano.oliveira@ccc.ufcg.edu.br} % email para contato
]
{
  \inits{KKMO}     % Letras iniciais do autor #4
  \fnm{Kaio Kassiano Moura} % Nome do autor #4 (first and middle name)
  \snm{Oliveira}     % Ultimo nome do autor #4 (last name)
}

%------------------------------------------------
% Endereço dos autores
%------------------------------------------------
\address[id=aff1]{
  \orgname{Universidade Federal de Campina Grande,
           Centro de Engenharia Elétrica e Informática,
           Unidade Acadêmica de Sistemas e Computação},
  \street{Rua Aprígio Veloso, 882, Bairro Universitário},
  \postcode{58429-140},
  \city{Campina Grande},
  \cny{Brasil.}
}

\end{fmbox}

%--------------------------------------------------------------------------------------------------
% Resumo do Trabalho
%--------------------------------------------------------------------------------------------------
\begin{resumobox}
	
\begin{resumo} 
Escrever no máximo $150$ palavras no resumo do trabalho.
\end{resumo}

%--------------------------------------------------------------------------------------------------
% Palavras-chaves: Entre 3 e 6 palavras chaves
%--------------------------------------------------------------------------------------------------
\begin{palavraschave}
  \kwd{Escreva}
  \kwd{algumas}
  \kwd{palavras-chaves}
  \kwd{aqui!}
\end{palavraschave}

\end{resumobox} % Não alterar

%--------------------------------------------------------------------------------------------------
% Abstract do Trabalho
%--------------------------------------------------------------------------------------------------
\begin{abstractbox}
	
\begin{abstract} 
Escrever no máximo $150$ palavras no resumo do trabalho.
\end{abstract}

%--------------------------------------------------------------------------------------------------
% Palavras-chaves: Entre 3 e 6 palavras chaves
%--------------------------------------------------------------------------------------------------
\begin{keyword}
  \kwd{Escreva}
  \kwd{algumas}
  \kwd{palavras-chaves}
  \kwd{aqui!}
\end{keyword}

\end{abstractbox} % Não alterar
\end{frontmatter} % Não alterar

%--------------------------------------------------------------------------------------------------
% Escreva o seu artigo!
%--------------------------------------------------------------------------------------------------

%------------------------------------------------
% Seção 1
%------------------------------------------------
\section{Introdução}
Texto

%------------------------------------------------
% Seção 2
%------------------------------------------------
\section{Motivação}
Texto

%------------------------------------------------
% Sub-seção 2.1
%------------------------------------------------
\subsection{Definição} %Definição de IoT
A Internet das Coisas é um conceito criado para a revolução tecnológica dos itens interconectados, 
com o objetivo de facilitar a realização de tarefas cotidianas.\\

“Mas o que eu quis dizer à época, e ainda considero isso válido, se baseia 
na ideia de que estamos presenciando o momento em que duas redes distintas – 
a rede de comunicações humana (exemplificada na internet) e o mundo real das coisas – 
precisam se encontrar. Um ponto de encontro onde não mais apenas "usaremos um 
computador", mas onde o "computador se use" independentemente, de modo a tornar a
vida mais eficiente. Os objetos – as "coisas" – estarão conectados entre si e em rede, 
de modo inteligente, e passarão a "sentir" o mundo ao redor e a interagir.”\\  

\textit{Kevin Ashton, pesquisador britânico do Massachusetts Institute of Technology(MIT), 
é considerado o primeiro especialista a usar o termo "Internet das Coisas" (IoT, na sigla em inglês), em 1999.}\\

Dessa forma, a IoT carrega a dinâmica e a interatividade às pessoas a partir 
da conexão e mobilidade dos dispositivos. 

%------------------------------------------------
% Sub-seção 2.2
%------------------------------------------------
\subsection{Surgimento do termo} %Surgimento de IoT
O termo “Internet das Coisas” é uma adaptação em português para a expressão em 
inglês “Internet of Things” (IoT, sigla em inglês), utilizada pela primeira 
vez pelo pesquisador britânico Kevin Ashton do Massachusetts Institute of 
Technology(MIT) no ano de 1999. Para Kevin, a revolução tecnológica iria 
proporcionar aos humanos uma maior independência dos aparelhos ao realizar 
tarefas sem a interferência direta humana.\\

“Numa apresentação para executivos da Procter & Gamble em 1999, quando eu 
falava da ideia de se etiquetar eletronicamente os produtos da empresa, 
para facilitar a logística da cadeia de produção, através de identificadores 
de radiofrequência (RFID, em inglês), na época um assunto novíssimo e "quente". 
A expressão "Internet das Coisas" pode nem ser tão brilhante, mas deu um bom 
título à apresentação, e logo se popularizou. Na verdade, a combinação de 
palavras foi como o resultado de um insight importante, de algo que ainda 
é mal compreendido.”\\

\textit{Kevin Ashton, pesquisador britânico do Massachusetts Institute of Technology(MIT).}\\

Um exemplo prático foi na preparação de atletas brasileiros na modalidade de canoagem, 
em parceria com a empresa multinacional GE desenvolveram um projeto de monitoramento 
dos movimentos dos esportistas, com objetivo de obter melhorias das performances nos 
treinos e, consequentemente, alcançar bons resultados nas Olimpíadas de 2016 na 
cidade do Rio de Janeiro. Este processo ocorreu através da captura dos dados de 
cada atleta no treinamento e enviados para a nuvem para técnicos analisarem os 
movimentos e terem o conhecimento da deficiência de cada atleta e trabalhar no 
aperfeiçoamento do rendimento de cada um.

%------------------------------------------------
% Sub-seção 2.3
%------------------------------------------------
\subsection{Como funciona}
Imagine uma situação, você entra cansado do trabalho, sua intenção 
é dormir,certo? Digamos que sua rotina geralmente é tomar um banho, 
jantar e depois dormir. Com a IoT, podemos espalhar sensores pela 
casa e com isso detectar seus movimentos do cotidiano e guardá-los 
em um banco de dados, dessa forma, os dispositivos que processam 
essas informações saberão sua rotina e passarão realizar tarefas 
e tomar decisões em situações do dia a dia.\\

Portanto, para a situação citada acima, os dispositivos conectados 
saberá a temperatura que você gosta de tomar banho. Após o banho, 
saberá que poderá jantar e sua geladeira ou armário informe se 
possui itens necessários para o preparo da comida. Após isso, 
saberá que talvez irá para o quarto e dormir, dessa forma irá 
deixar o quarto na temperatura ideal, a televisão ligará no canal 
mais assistido ou do programa que mais gosta até que durma e seu 
abajur e televisão desligue.\\

Mas como isso é possível? A conectividade é a palavra chave para o 
funcionamento dos aparelhos com o princípio de Internet das Coisas, 
logo torna possível a conexão inteligente entre os dispositivos e 
serem capazes de tomar decisões para a melhoria e conforto de 
atividades realizadas no cotidiano, seja ela como enviar mensagens 
quando não podemos atender, seja avisar que sua dispensa está ficando vazia.

%------------------------------------------------
% Seção 3
%------------------------------------------------
\section{Impactos da Internet das Coisas no cotidiano}
TEXTO INTRODUZINDO

%------------------------------------------------
% Sub-seção 3.1
%------------------------------------------------
\subsection{Vantagens do uso da IoT}
TEXTO

%------------------------------------------------
% Sub-seção 3.2
%------------------------------------------------
\subsection{Os riscos oferecidos com o uso da IoT}
Com toda essa inovação tecnológica, vários componentes interconectados 
e interagindo entre si facilitando a vida das pessoas surgiu questões 
sobre os possíveis riscos que a Iot pode causar, imagine utilizar algum 
objeto, seja ele celular, relógio, óculos e etc, para se comunicar com 
outra pessoa e alguém têm acesso a elas. Um  cenário preocupante, pois 
ao mesmo tempo que tem-se muita facilidade, também há preocupações com 
a segurança da informação.\\

Todas as criações que ocorrem na IoT podem ser distribuídos em 
dispositivos que coletam informações através de sensores do ambiente 
para transmitir informação ou em dispositivos desenvolvidos para receber
instruções via Internet e realizarem alguma atividade. Contudo não 
se sabe acerca de quem tem acesso às informações dos dispositivos e 
quem comanda as ações destes. Por exemplo, os veículos autônomos, como 
carros e drones, dependem de sensores para funcionar e um hacker pode 
obter controle total de um carro por meio de redes sem fio e enganar 
vários tipos de sensores para mostrar ao motorista dados falsos.\\

Outro fato importante é que os roteadores são os alvos principais dos 
hackers que desejam ter acesso a uma rede específica. A IoT realiza 
conexões através de endereços de IP dos dispositivos,  essa conexão é 
muito flexível, pois é possível criar senha para protegê-las mas nem 
sempre isso funciona.\\

A Nexusguard, empresa de soluções de segurança para o combate a 
Distributed Denial of Service (ataques distribuídos de negação de 
serviço), realizou um estudo que examina os riscos de dispositivos 
sempre conectados. O levantamento foi conduzido pela empresa de pesquisa 
e de inteligência de mercado Cybersecurity Ventures.Terrence Gareau, 
cientista-chefe da Nexusguard, afirma que “Os roteadores domésticos e 
outros dispositivos similares conectados à internet são pontos fáceis 
de acesso para os hackers, que podem utilizá-los para causar um DDoS ou 
estabelecer proxies para golpes na Internet que podem derrubar provedores 
de acesso ou prejudicar uma empresa. Esses ataques podem ser particularmente 
nocivos para os provedores de serviços de IoT, em um sistema de alarme 
que é controlado por um aplicativo, por exemplo. Um ataque pode desligar 
totalmente esse recurso. Somos o principal player na prevenção de 
ataques de DDoS e a serviços de IoT e acreditamos que é importante 
sensibilizar a indústria sobre as ameaças persistentes de IoT”.

%------------------------------------------------
% Seção 4 
%------------------------------------------------
\section{Considerações finais}
FALAR SOBRE A EXPECTATIVAS E DESAFIOS PARA O FUTURO


%--------------------------------------------------------------------------------------------------
% FIM DO ARTIGO
%--------------------------------------------------------------------------------------------------

%--------------------------------------------------------------------------------------------------
%--------------------------------------------------------------------------------------------------
% Define o arquivo BIB (bibliografia)
%--------------------------------------------------------------------------------------------------
%--------------------------------------------------------------------------------------------------
\bibliographystyle{bmc-mathphys}   % NAO EDITAR!
\bibliography{artigo_bibliografia} % NAO EDITAR! - Bibliography file (usually '*.bib' )

\vspace{1.0cm}
\noindent {\bf Alice Fernandes Silva}

\noindent {\bf Ellen Oliveira}


\noindent {\bf Ivyna Rayany Santino Alves}


\noindent {\bf Kaio Kassiano Moura Oliveira}
\end{document}
